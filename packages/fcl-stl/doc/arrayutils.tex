\chapter{TArrayUtils}

Set of utilities for manipulating arrays data.

Takes 3 arguements for specialization. First one is type of array (can be anything, which is
accesible by [] operator, e. g. ordinary array, vector, ...), second one is type of array element.

%Usage example for sorting:

%\lstinputlisting[language=Pascal]{sortingexample.pp}

Members list:

\begin{longtable}{|m{10cm}|m{5cm}|}
\hline
Method & Complexity guarantees \\ \hline
\multicolumn{2}{|m{15cm}|}{Description} \\ \hline\hline

\verb!procedure RandomShuffle(arr: TArr, size:SizeUint)! &
O(N)\\ \hline
\multicolumn{2}{|m{15cm}|}{Shuffles elements in array in random way} \\\hline\hline

\end{longtable}\chapter{TOrderingArrayUtils}

Set of utilities for manipulating arrays data.

Takes 3 arguements for specialization. First one is type of array (can be anything, which is
accesible by [] operator, e. g. ordinary array, vector, ...), second one is type of array element,
third one is comparator class (see TPriorityQueue for definition of comparator class).

Usage example for sorting:

\lstinputlisting[language=Pascal]{sortingexample.pp}

Members list:

\begin{longtable}{|m{10cm}|m{5cm}|}
\hline
Method & Complexity guarantees \\ \hline
\multicolumn{2}{|m{15cm}|}{Description} \\ \hline\hline

\verb!procedure Sort(arr: TArr, size:SizeUint)! &
O(N log N) average and worst case. Uses QuickSort, backed up by HeapSort, when QuickSort ends up in
using too much recursion.\\ \hline
\multicolumn{2}{|m{15cm}|}{Sort array arr, with specified size. Array indexing should be 0 based.} \\\hline\hline

<<<<<<< HEAD
<<<<<<< HEAD
<<<<<<< HEAD
\verb!function NextPermutation! \verb!(arr: TArr, size:SizeUint):boolean! &
Worst case for one call $O(N)$. Going through all permutations takes $O(N!)$ time.\\ \hline
\multicolumn{2}{|m{15cm}|}{Orders elements on indexes $0, 1, \dots, size-1$ into nearest
lexikografic larger permutation.} \\\hline


=======
>>>>>>> graemeg/cpstrnew
=======
>>>>>>> graemeg/cpstrnew
=======
>>>>>>> graemeg/cpstrnew
\end{longtable}
